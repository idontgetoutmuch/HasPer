\documentclass{article}

\usepackage{listings}
\usepackage{a4}
\usepackage{courier}
\usepackage{hyperref}
\usepackage{html}


\lstdefinelanguage{ASN1} {
  morekeywords={},
  sensitive=false,
  morecomment=[s]{(--}{--)}
  }

\lstdefinelanguage{shell} {
  sensitive=true
  }

\setlength{\parskip}{\medskipamount}
\setlength{\parindent}{0pt}

\title{Haskell ASN.1 PER Library 1.0.0}
\author{Dominic Steinitz}

\begin{document}

\maketitle

The 
\htmladdnormallinkfoot
{Haskell Cryptographic Library 1.0.0}
{http://www.haskell.org/asn1}
is an implementation of the ASN.1 Packed Encoding Rules (PER).

The current release does not contain a parser for ASN.1 so you have to hand encode
ASN.1 definitions into the Haskell ASN.1 Abstract Syntax Tree (AST). Examples of these are given
for each ASN.1 construction supported.

Haddock documentation for the library is available
\htmladdnormallinkfoot
{here}
{http://www.haskell.org/asn1/doc/html};
alternatively, you can generate it yourself (need to give instructions on how to do this).

The library has been designed to prevent encoding errors at compile time. However, given the nature
of ASN.1 and Haskell, it is not possible to do this in all cases.

The library comes with:

\begin{enumerate}

\item
Tests for each supported type ({\em UnitTest.lhs}).

\item
Properties which are tested
by
\htmladdnormallinkfoot
{QuickCheck}
{http://www.cs.chalmers.se/~rjmh/QuickCheck/}.

\item
Tests for the examples given in
\htmladdnormallinkfoot
{ASN.1 Complete}
{http://www.oss.com/asn1/larmouth.html}
by John Larmouth.

\item
Tests for the examples given in
\htmladdnormallinkfoot
{ASN.1 --- Communication between heterogeneous systems}
{http://asn1.elibel.tm.fr/en/book/index.htm}
by Olivier Dubuisson.

\item
Tests for the examples given in Annex A of
\htmladdnormallinkfoot
{Information technology --- ASN.1 encoding rules: Specification of Packed Encoding Rules (PER)}
{http://www.itu.int/rec/T-REC-X.680-X.693/e}
also known as X.691 (07/2002).

\end{enumerate}

Furthermore, the library also comes with properties tested by QuickCheck of inter-operation with
\htmladdnormallinkfoot
{the open source ASN.1 compiler for C}
{http://lionet.info/asn1c}
(asn1c) subject to asn1c supporting the types supported in the Haskell
library. In other words, arbitrary ASN.1 types and values are generated using QuickCheck,
encoded with asn1c, decoded with the Haskell library and then checked
to ensure that the value decoded in Haskell is the same as the value
encoded in C (and vice versa).

\section{ASN.1 Support}

\subsection{BOOLEAN}

\lstset{language=Haskell}

\begin{lstlisting}[frame=single]
BOOLEAN :: ASNType Bool

falseEncoding = toPer BOOLEAN False
\end{lstlisting}

\subsection{INTEGER}

\begin{lstlisting}[frame=single]
INTEGER :: ASNType Bool

integerEncoding = toPer INTEGER 4096
\end{lstlisting}

\subsection{ENUMERATED}

\subsection{REAL}

We don't currently support this. Do we want to?

\subsection{BIT STRING}

\begin{lstlisting}[frame=single]
BITSTRING :: NamedBits -> ASNType BitString

encodedBitString = 
   toPer (BITSTRING []) 
         (BitString [1,1,0,0,0,1,0,0,0,0])
\end{lstlisting}

We don't seem to have any named bits tests or examples.

\subsection{CHOICE}

\begin{lstlisting}[frame=single]

FooBaz {1 2 0 0 6 3} DEFINITIONS ::=
   BEGIN
      Choice1 ::= 
        CHOICE {
          d INTEGER,
          e BOOLEAN
	}
   END
\end{lstlisting}

\begin{lstlisting}[frame=single]

tooMany = 3 :+: (True :+: ASNEmpty)
tooFew = NoValue :-: (NoValue :-: ASNEmpty)
justRight1 = NoValue :-: (True :+: ASNEmpty)

\end{lstlisting}

\section{Installation Instructions}

Get the sources:

\lstset{language=shell,basicstyle=\ttfamily\small}
\begin{lstlisting}[frame=single]
darcs get --tag "1.0.0" http://darcs.haskell.org/asn1
\end{lstlisting}

Build and install ready for testing:

\begin{lstlisting}[frame=single]
ghc -o Setup Setup.hs -package Cabal
./Setup configure --prefix=/my/chosen/dir
./Setup build
./Setup install --user
\end{lstlisting}

Run the tests.

\begin{lstlisting}[frame=single]
cd /my/chosen/dir/bin
./UnitTest
\end{lstlisting}

You can now run the examples to confirm further that everything
is working satisfactorily.
When you are happy, build and install them in
their final destination:

\begin{lstlisting}[frame=single]
./Setup unregister --user
./Setup clean
./Setup configure
./Setup build
./Setup install
\end{lstlisting}

\section{To Do}

In no particular order:

\begin{itemize}
\item
Parser.
\item
Object classes.
\end{itemize}

\section{Contact}

All questions, comments, bug reports, flames, requests for 
updates / changes and suggestions should be directed to Dominic Steinitz and
logged
\htmladdnormallinkfoot
{here}
{http://code.google.com/p/hasn1/}
.


The modules in the library come from different authors and have been 
released under different licences. 

\section{Licensing}

This license is based on
\htmladdnormallinkfoot
{The BSD License}
{http://www.opensource.org/licenses/bsd-license.php}.

Redistribution and use in source and binary forms, with or without 
modification, are permitted provided that the following conditions are met:

\begin{itemize}
\item
Redistributions of source code must retain the above copyright notice, 
this list of conditions and the following disclaimer.
\item
Redistributions in binary form must reproduce the above copyright notice, 
this list of conditions and the following disclaimer in the documentation 
and/or other materials provided with the distribution.
\item
The names of its contributors may not be used to endorse or promote 
products derived from this software without specific prior written permission.
\end{itemize}

\begin{sc}
This software is provided by the copyright holders and contributors ``AS IS'' 
and any express or implied warranties, including, but not limited to, 
the implied warranties of merchantability and fitness for a particular 
purpose are disclaimed. In no event shall the copyright onwers or
contributors be liable for any direct, indirect, incidental, special,
exemplary, or consequential damages (including, but not limited to,
procurement of substitute goods or services; loss of use, data, or profits;
or business interruption) however caused and on any theory of liability,
whether in contract, strict liability, or tort (including negligence or
otherwise) arising in any way out of the use of this software,
even if advised of the possibility of such damage.
\end{sc}

\section{Contributors}

\begin{itemize}

\item
\htmladdnormallinkfoot
   {Dan Russell}
   {http://www.kingston.ac.uk/~ku02309/}

\item
Dominic Steinitz

\end{itemize}

This document was last updated on 4th November 2007.
\copyright\ 2006--2007 Dominic Steinitz. 

\end{document}
